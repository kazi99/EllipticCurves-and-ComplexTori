\documentclass[mat1]{fmfdelo}
% \documentclass[fin1]{fmfdelo}
% \documentclass[isrm1]{fmfdelo}
% \documentclass[mat2]{fmfdelo}
% \documentclass[fin2]{fmfdelo}
% \documentclass[isrm2]{fmfdelo}

% naslednje ukaze ustrezno napolnite
\avtor{Izak Jenko}

\naslov{Kompleksni torusi in eliptične krivulje}
\title{Complex tori and elliptic curves}

% navedite ime mentorja s polnim nazivom: doc.~dr.~Ime Priimek,
% izr.~prof.~dr.~Ime Priimek, prof.~dr.~Ime Priimek
% uporabite le tisti ukaz/ukaze, ki je/so za vas ustrezni
\mentor{izr.~prof.~dr.~Sašo Strle}
% \mentorica{}
% \somentor{}
% \somentorica{}
% \mentorja{}{}
% \mentorici{}{}

\letnica{2021} % leto diplome

%  V povzetku na kratko opišite vsebinske rezultate dela. Sem ne sodi razlaga organizacije dela --
%  v katerem poglavju/razdelku je kaj, pač pa le opis vsebine.
\povzetek{}

%  Prevod slovenskega povzetka v angleščino.
\abstract{}

% navedite vsaj eno klasifikacijsko oznako --
% dostopne so na www.ams.org/mathscinet/msc/msc2020.html
\klasifikacija{}
\kljucnebesede{} % navedite nekaj ključnih pojmov, ki nastopajo v delu
\keywords{} % angleški prevod ključnih besed

\zapisiMetaPodatke  % poskrbi za metapodatke in veljaven PDF/A-1b standard

% aktivirajte pakete, ki jih potrebujete
% \usepackage{tikz}
\usepackage{bm}

% za številske množice uporabite naslednje simbole
\newcommand{\R}{\mathbb R}
\newcommand{\N}{\mathbb N}
\newcommand{\Z}{\mathbb Z}
\newcommand{\C}{\mathbb C}
\newcommand{\Q}{\mathbb Q}

% matematične operatorje deklarirajte kot take, da jih bo Latex pravilno stavil
% \DeclareMathOperator{\conv}{conv}

% vstavite svoje definicije ...
%  \newcommand{}{}

\begin{document}

%%%%%%%%%%%%%%%%%%%%%%%%%%% 1. UVOD %%%%%%%%%%%%%%%%%%%%%%%%%%%%%%%%%%%

\section{Uvod}

Matematike pogosto zanimajo rešitve različnih enačb. Obstoj rešitev, kakšne lastnosti imajo in
kako se obnašajo pod raznimi transformacijami. Osrednja tema moje naloge bo preučiti in ustvariti
geometrijsko predstavo množice ničel kompleksnega polinoma tretje stopnje posebne oblike. 
To množico ničel si lahko predstavljamo kot realno ploskev in ji pravimo eliptična krivulja.   
Zgodovinsko je eliptična krivulja množica ničel enačbe
$$ y^2 = x^3 + ax + b. $$
V tem delu pa se bomo ukvarjali z nekoliko prilagojeno -- projektivno -- obliko te enačbe. Množicam ničel
polinomov več spremenljivk pravimo \emph{algebraične krivulje} in z njimi bomo začeli v poglavju \ref{algebraicne krivulje}. 
\par 
Pri iskanju rešitev
polinomskih enačb, se razmeroma hitro porodi vprašanje, iz katerega ambientnega prosotra 
sploh sprejmemo veljavne rešitve. Spomnimo se fundametalnega izreka algebre, ki pravi, da ima
vsak nekonstanten polinom s kompleksnimi koeficienti ničlo v polju kompleksnih števil, med tem
ko brez težav poiščemo realne polinome, ki realnih ničel nimajo. Podobno situacijo imamo tukaj. 
Eliptične krivulje se namreč da študirati nad mnogo različnimi polji. Nad končnimi polji igrajo
eliptične kriulje pomembno vlogo v kriprografiji, nad racionalnimi števili v algebraični teoriji
števil, mi pa jih bomo v tem delu gledali nad poljem kompleksnih števil. 
\par
V primeru obravnave nad poljem kompleksnih števil eliptične
krivulje naravno pridobijo dodatno kompleksno stukturo in na ta način postanejo t.i.\ \emph{Riemannove ploskve}. Ta struktura nam omogoča analizo holomorfnih funkcij na prostorih, ki niso nujno domene v kompleksni ravnini in jo
bomo bolj podrobno preiskali v poglavju \ref{riemannove ploskve}. V nadaljevanju bomo videli, da tudi torus premore strukturo Riemannove ploskve in ga bomo skupaj s to strukturo imenovali kompleksni torus. Izkaže se, da je kompleksni
torus najbolj smiselna domena dvojno periodičnih oz. eliptičnih funkcij. Lastnosti in obnašanje eliptičnih funkcij si bomo ogledali v poglavju \ref{elipticne funkcije}, klučno vlogo pa bo igrala prav posebna Weierstrassova eliptična funkcija $\wp$. Ta nam bo nazadnje v poglavju \ref{poglavje izomorfizem} omogočila konstrukcijo preslikave, ki bo pokazala, da sta kompleksni torus in eliptična krivulja v nekem smislu enaka matematična objekta.

Vredno je še opomniti, da eliptične krivulje in področja v katerih se uporabljajo 
nimajo več vsebinsko praktično nič opravka z elipsami. 
Izkazalo se je, da so inverzi funkcij s katermi računamo dolžine lokov elips, dvojno periodični oz. eliptični,
če jih gledamo kot funkcije kompleksne spremenljivk. Te dvojno periodične funkcije pa so tesno povezane z enačbo,
ki ji zadoščajo eliptične krivulje in se bomo k njim vrnili v poglavju \ref{elipticne funkcije}.

%%%%%%%%%%%%%%%%%%%%%% 2. ALGEBRAIČNE KRIVULJE %%%%%%%%%%%%%%%%%%%%%%%%%%%%%

\section{Algebraične krivulje} \label{algebraicne krivulje}
Algebraične krivulje so množice ničel polinomov nad različnimi polji. V tem poglavju bomo začeli 
z afinimi
algebraičnimi krivuljami, ki jih v nadaljevanju sicer ne bomo direktno potrebovali, bodo pa igrale pomembno
vlogo pri razumevanju projektivnih algebraičnih krivulj, ki jih bomo vpeljali takoj za tem. 
Zaradi namenov tega dela, algebraičnih krivulj ne bomo obravnavali nad povsem splošnimi polji, pač pa se
bomo omejili na polje kompleksnih števil, ki ga bomo označevali s $\C$. V smislu enodimenzionalnega kompleksnega prostora, bomo množici kompleksnih števil pravili tudi kompleksna premica.

%%%%% Afine algebraične krivulje %%%%%

\subsection{Afine algebraične krivulje} 
Naj $\C[x_1, \dots x_n]$ označuje kolobar polinomov $n$ spremenljivk s
kompleksnimi koeficienti. Množica ničel poljubnega polinoma $f \in \C[x_1, \dots x_n]$ je
$$ V(f) = \{p \in \C^n \mid f(p) = 0 \} \subseteq \C^n.$$

\begin{definicija}
Množica $C \subseteq \C^2$ je \emph{afina algebraična krivulja}, če obstaja tak polinom $f \in \C[x,y]$ stopnje vsaj $1$, da je
\begin{equation*}
    C = V(f).
\end{equation*}
\end{definicija}

Afine algebraične krivulje si lahko predstavljamo, kot nekaj podobnega ploskvam v prostoru $\R^4$, če naredimo identifikacijo $\C \equiv \R^2$. Dve kompleksni spremenljivki polinoma lahko zamenjamo s štirimi realnimi, prav tako pa tedaj tudi polinomska enačba $f(x,y) = 0$ razpade na dve realni. To sta
\begin{equation*}
    \Re f(x_1 + ix_2, y_1 + iy_2) = 0 \quad \text{in} \quad \Im f(x_1 + ix_2, y_1 + iy_2) = 0,
\end{equation*}
kjer so $x_1, x_2, y_1, y_2 \in \R$ realne spremenljivke.
Pogoji, ki jim zadoščajo točke na afini algebraični krivulji $C \subseteq \R^4$, so zelo podobni tistim, ki definirajo gladke pod\-mnogoterosti z glavno razliko, da gradienti teh definicijskih funkcij niso nujno (realno) linearno neodvisni. To bi bilo na $C$ razvidno kot samopresečišča ali osti, ki pa jih podmnogoterosti seveda nimajo. 
\par
V ta namen bi radi definirali singularne točke na afini algebraični krivulji $C = V(f)$ kot rešitve sistema enačb $\nabla f(x_0, y_0) = 0$, $f(x_0, y_0) = 0$. Toda ta definicija zaenkrat ni dobra, saj polinom $f \in \C[x,y]$ ni enolično določen s krivuljo $C$. Zato najprej uvedemo pojem minimalnega polinoma krivulje $C$.

% V definiciji, bi radi uporabili polinom $f \in \C[x,y]$, katerega množica ničel je krivulja $C$, toda ta poliom s krivuljo ni enolično določen zato najprej uvedimo pojem minimalnega polinoma krivulje $C$.
\begin{primer}
   demonstracija zakaj je pomembno uvesti minimalni polinom.
\end{primer}

\begin{definicija}
Naj bo $C$ afina algebraična krivulja. \emph{Minimalni polinom} krivulje $C$ je polinom $f \in C[x,y]$ najmanjše stopnje, za katerega velja $V(f) = C$.
\end{definicija}

S pomočjo minimalnega polinoma krivulje, lahko sedaj definiramo singularne in regularne točke na njej.

\begin{definicija}
    Naj bo $C$ afina algebraična krivulja in $f \in \C[x,y]$ njen minimalni polinom. Točka $(x_0, y_0) \in C$ je \emph{regularna}, če velja
    $$\frac{\partial f}{\partial x}(x_0, y_0) \neq 0 \quad \text{ ali } \quad \frac{\partial f}{\partial y}(x_0, y_0) \neq 0,$$ 
    in \emph{singularna} sicer.
\end{definicija}

To nam omogoči formulirati prvo opazko.

\begin{trditev}
Vsaka afina algebraična krivulja $C \subseteq \C^2$ brez singularnih točk je z identifikacijo $\C^2 \equiv \R^4$ gladka $2$-podmnogoterost oz. ploskev.
\end{trditev}

\begin{dokaz}
    pozneje.
\end{dokaz}

Ta trditev pove katere od afinih algebraičnih krivulj ne le lokalno v okolici regularnih točk izgledajo kot ploskve, temveč tudi so zares ploskve. 
\par
Na tem mestu se pojavi manjša nejasnost, zakaj afine algebraične krivulje poimenovati ravno \emph{krivulje}. V kontekstu realnih podmnogoterosti se sprva to poimenovanje res zdi malce neusklajeno, toda v okviru kompleksnih dimenzij ta terminologija postane smiselna. Če v definiciji podmnogoterosti namreč samo zamenjamo polje realnih števil s $\C$ se povedano bistveno ne spremeni. Še vedno ohranimo dejstvo, da število "linearno neodvisnih" enačb ustreza kodimenziji podmnogoterosti in analogno tudi dimenzija podmnogoterosti ustreza razliki (kompleksne) dimenzije ambientnega prostora in kodimenzije. V tem smislu so potem ti objekti, ki jih realno vidimo kot ploskve, zares tudi kompleksne $1$-podmnogoterosti oziroma krivulje.

% \begin{opomba}
% Poimenovati te objekte ravno krivulje, se morda v tem kontekstu na prvi pogled zdi malce neusklajeno z našo predstavo iz realnih evlidskih prostorov. Toda, ker jih obravnavamo v smislu kompleksnih dimenzij 
% \end{opomba}
 
%%%%% Projektivne algebraične krivulje %%%%%

\subsection{Projektivne algebraične krivulje}

V tem razdelku bomo algebraične krivulje krivulje obravnavali še v projektivnem smislu. Definirali bomo kompleksno projektivno ravnino in krivulje na njej. Vpeljavo projektivne ravnine opravičujemo z mnogimi lepimi lastnostmi v povezavi s presečišči krivulj v njej, pa tudi z raznimi bolj topološkimi razlogi, kot so na primer kompaktnost algebraičnih krivulj.

Najprej bomo obravnavali kompleksno projektivno ravnino in njene lastnosti. Glavna ideja za konstrukcijo 
\begin{definicija}
\emph{Kompleksen projektiven prostor} dimenzije $n$, je 
\begin{equation*}
    P^n(\C) = (\C^{n+1} \setminus \{\bm0 \}) / \langle v \sim \lambda v; \lambda \in \C^{\times} \rangle
    %/ \langle v \sym \lambda v; \lambda \in \C^{\times}
\end{equation*}
\end{definicija}



\subsection{Nesingularne kubike}
\break

\section{Eliptične funkcije} \label{elipticne funkcije}
\subsection{Lastnosti eliptičnih funkcij}
\subsection{Weierstrassova funkcija $\wp$}
\break

\section{Riemannove ploskve} \label{riemannove ploskve}
\subsection{Definicija in lastnosti}
\subsection{Kompleksna struktura na eliptični krivulji}
\subsection{Kompleksna struktura na torusu}
\break

\section{Izomorfizem \texorpdfstring{$\C/\Lambda \to E(\C)$}{} in njegove posledice} \label{poglavje izomorfizem}
\break

\section*{Slovar strokovnih izrazov}

\geslo{}{}
\geslo{}{}

% seznam uporabljene literature
\begin{thebibliography}{99}

%\bibitem{}

\end{thebibliography}

\end{document}

